\documentclass{llncs}
\usepackage{graphicx}
\usepackage{geometry}
\geometry{margin=2cm}
\usepackage{mathtools}
\usepackage{listings}

\begin{document}
\title{Contribution Title\thanks{Supported by organization x.}}
\author{Xi, Du\inst{1}}
\institute{Australian National University, Australia}
\maketitle 
\begin{abstract}
While the recent surge of neural networks in the name of deep learning literally suggests larger and larger networks, 
small networks have also received growing attention lately. 
So has the technique to make a trained network smaller while keeping its inference ability,
which is called network pruning or network reduction. 
In this work, I briefly review some of the network reduction techniques and 
experiment with one of the early methods \cite{method} using 
a data set about human eye movement during reading \cite{data}

\keywords{Neural network, Deep learning, 
Small data, Interpretability, Network pruning, Network reduction.}
\end{abstract}

\section{Introduction}

Deep as in deep learning,
the word leading the current resurgence of neural networks in machine learning 
and artificial intelligence,
intrinsically means larger-than-before neural networks with more and more layers and,
frequently but not necessarily,
more neurons in each layer.
However,
larger networks not only consume more computational resource during training and inference,
but are also harder to understand,
analyze and debug when problems arise.
Most importantly,
overly complex networks usually lead to bad generalisation power.
So given similar predictive power,
it is desirable to use a smaller network whenever possible.
In addition,
although neural networks are said to be bad at this in general,
smaller networks are still better in terms of interpretability,
an emerging topic in machine learning \cite{interp}.

\section{Literature Review}

The first significant work to bring the size of a trained neural network 
down was probably \cite{prune0},
where the word �pruning� was used and started being the most widely used terminology
for making a trained neural network smaller. 
However, a drawback of this word is its connotations of simply removing neurons 
without changing the remaining ones, 
which is less interesting than changing weights of neurons so that they work equivalently 
to a larger network. 
This was probably the reason the method we based on was named network reduction \cite{method}.

\section{Method}

\subsection{Preliminary}

First of all, my implementation and experimental setup 
are concerned about a 3-layer network of a single 
hidden layer of $H$ neurons. In the following discussion 
$N$ stands for the number of entries in the training set.

\subsection{Implementation of the Network Reduction Technique}

The network reduction paper \cite{method} mentioned many ways in which 
networks can be pruned. 
However, only two ways were discussed in detail. 
The first was considering whether a group of hidden layer units 
can be removed or compressed into one. 
Unfortunately, a straightforward implementation would require examining each 
power set of hidden units, to the complexity of $O(2^H)$. 
This is not feasible for any practically-sized neural network, 
so I focused on the second technique, {\em Distinctiveness}.

The {\em Distinctiveness} method examines each pair of hidden units, 
and then combine them into one when they are similar enough in terms of 
outputs of the training set, resulting in $O(H^2)$ tests, 
which is much more manageable. 
While there may be groups of three or more hidden units that can be collapsed into one, 
they are already catered by the $O(n^2)$ tests. 
For example, if hidden unit A and hidden unit B were deemed similar 
enough to be combined into one, 
and then hidden unit A and hidden unit C were deemed similar enough as well, 
my implementation would remove B at first, then remove C, 
adding weights to A accordingly. 

The similarity between hidden units are established by comparing the angles 
between the vectors of hidden unit outputs. 
More specifically, a vector of the output of the hidden unit $A$, 
is just the outputs of $A$ across training data point. 
In my implementation, 
these $N$-dimension vectors are computed all together at once into an $N\times H$ matrix, 
avoiding an explicit loop. 
Pythonic loops in PyTorch are extremely expensive and should be avoided whenever possible.
While the original paper listed actual values of vector angles, 
there is no real need to actually bother with an inverse trigonometry function. 
Only the cosine is necessary, since I can threshold on the cosine value directly, 
which is
\begin{equation}
\cos(\theta) = \frac{\vec{a}\cdot \vec{b} }{\lVert \vec{a}\rVert \lVert \vec{b}\rVert }
\end{equation}
where $\vec{a}$ and $\vec{b}$ are $N$-dimensional vectors.
Because there are $H$ such vectors,
we can consider the $H$ $N$-dimensional vectors together
as one $H\times N$ matrix $h^T$,
where its $N\times H$ transpose $h$ happens to be
a natural occurence when training
neural network with PyTorch.
Then the $H^2$ dot products are computed with one $H\times N$ matrix
\begin{equation}
	P=h^Th~.
\end{equation}
The matrix $L$ of $H^2$ numerators can be computed in a similar fashion with
matrix operations instead of explicit loops.
Finally an $H\times H$ matrix $C$ of cosine values are computed 
by an element-wise divison of $P$ over $L$.
Note that $P$, $L$ and $C$ are all $H\times H$ symmetric matrices.
While the the cosines between different points are computed twice this way, 
it is still far more efficient than an explicit pythonic loop.

Now that I have the matrix of cosines, I have to run an explicit loop
to prune the hidden units.
However, the pruning happens only after many cycles of training,
so the cost is not a major issue compared to that of training.
For each pair of hidden units $i$ and $j$ ($0<i<j<=H$ and $i,j$ actually stands
for indices of hidden units), 
when the absolute value of the cosine of their angle, that is $|C_{i,j}|$, is greater
than a certain threshold $c$ where $0<=c<=1$, 
the $j$th hidden unit is removed.
In practice, I simply set its weight into the output layer to zero.
The weight is multipled by the cosine value $C_{i,j}$ 
and added to the out-going weight of $i$th hidden unit.
Of course, in theory either $i$ or $j$ can be kept,
but there is no obvious way to prefer one. 
Randomisation is not necessary either, because the
order of hidden units is already random in a sense.
Multiplication by $C_{i,j}$ automatically takes care
of the sign of the $C_{i,j}$. It may also be
argued that multiplication by $C_{i,j}$ is 
preferrable than simple addition and subtraction
because the angle is not strictly $0^\circ$ or $180^\circ$.
But I am already assuming that $|C_{i,j}|\simeq 1$, so it does not really matter.

An additional condition for the pruning is that hidden unit $i$
has not already been pruned, that is, its outgoing weight is not strictly zero.
This is an exact floating point equality comparison
arguably against common coding practice.
The rationale is that I am using the exact zero as the flag for having been pruned.
Even if a node happens to have zero outgoing weight immediately after training,
I lose no accuracy by treating it as having been pruned.
On the other hand, if I set the outgoing weight of hidden unit $j$ to zero
and move the weight, I fail to actually reduce the number of hidden units.

\subsection{Implementation of Neural Network}

The neural network was implemented with PyTorch.
Since the network is simple, I used 
straightforward matrix (or tensor in PyTorch jargon) computations and autograd
with very little content from \verb|torch.nn|.

\subsubsection{Training}

In the training phase,
The input is treated as an $N\times D_{in}$ matrix $X$.
$D_{in}$ is the dimensionality of the data, effectively $4$ for 
the eye-gaze dataset \cite{data}.
Weights from the input layer to the hidden layer is stored 
in an $D_{in} \times H$ matrix $W_1$.
Then obviously $XW_1$ becomes the inputs of activation functions.
The activation functions are the same for the $H$ hidden units.
However, I have made the activation function configurable in the program,
enabling comparison of networks with different activation functions.
The dimensionality of output, $D_{out}$ is 1 because this is a
classification test.
The weights from the hidden layer to the output layer form a 
$H \times D_{out}$ matrix $W_2$.
The outputs of the training set are
an $H \times D_{out}$ matrix $Y$
while the predicted outputs are $Y_{pred}$ of the same dimension.
The prediction is computed by
\begin{equation}
Y_{pred} = hW_2~.
\end{equation}
Finally the back-propogation is handled by standard PyTorch
autograd facility.

\subsubsection{Testing}

The testing phase is handled by exactly the same matrix operations
of the training phase,
albeit the dimensions of input and output matrices have $N'$ rows instead of $N$ rows,
where $N'$ is the size of the test set.
The autograd facility of PyTorch needs to be temporarily disabled with
\verb|torch.no_grad()| when executing the testing phase.

\subsection{Experiments}

\section{First Section}
\subsection{A Subsection Sample}
Please note that the first paragraph of a section or subsection is
not indented. The first paragraph that follows a table, figure,
equation etc. does not need an indent, either.

Subsequent paragraphs, however, are indented.

\subsubsection{Sample Heading (Third Level)} Only two levels of
headings should be numbered. Lower level headings remain unnumbered;
they are formatted as run-in headings.

\paragraph{Sample Heading (Fourth Level)}
The contribution should contain no more than four levels of
headings. Table~\ref{tab1} gives a summary of all heading levels.

\begin{table}
\caption{Table captions should be placed above the
tables.}\label{tab1}
\begin{tabular}{|l|l|l|}
\hline
Heading level &  Example & Font size and style\\
\hline
Title (centered) &  {\Large\bfseries Lecture Notes} & 14 point, bold\\
1st-level heading &  {\large\bfseries 1 Introduction} & 12 point, bold\\
2nd-level heading & {\bfseries 2.1 Printing Area} & 10 point, bold\\
3rd-level heading & {\bfseries Run-in Heading in Bold.} Text follows & 10 point, bold\\
4th-level heading & {\itshape Lowest Level Heading.} Text follows & 10 point, italic\\
\hline
\end{tabular}
\end{table}


\noindent Displayed equations are centered and set on a separate
line.
\begin{equation}
x + y = z
\end{equation}
Please try to avoid rasterized images for line-art diagrams and
schemas. Whenever possible, use vector graphics instead (see
Fig.~\ref{fig1}).

\begin{theorem}
This is a sample theorem. The run-in heading is set in bold, while
the following text appears in italics. Definitions, lemmas,
propositions, and corollaries are styled the same way.
\end{theorem}
%
% the environments 'definition', 'lemma', 'proposition', 'corollary',
% 'remark', and 'example' are defined in the LLNCS documentclass as well.
%
\begin{proof}
Proofs, examples, and remarks have the initial word in italics,
while the following text appears in normal font.
\end{proof}
For citations of references, we prefer the use of square brackets
and consecutive numbers. Citations using labels or the author/year
convention are also acceptable. The following bibliography provides
a sample reference list with entries for journal
articles~\cite{ref_article1}, an LNCS chapter~\cite{ref_lncs1}, a
book~\cite{ref_book1}, proceedings without editors~\cite{ref_proc1},
and a homepage~\cite{ref_url1}. Multiple citations are grouped
\cite{ref_article1,ref_lncs1,ref_book1},
\cite{ref_article1,ref_book1,ref_proc1,ref_url1}.
%
% ---- Bibliography ----
%
% BibTeX users should specify bibliography style 'splncs04'.
% References will then be sorted and formatted in the correct style.
%
% \bibliographystyle{splncs04}
% \bibliography{mybibliography}
%
\begin{thebibliography}{8}
\bibitem{ref_article1}
Author, F.: Article title. Journal \textbf{2}(5), 99--110 (2016)

\bibitem{ref_lncs1}
Author, F., Author, S.: Title of a proceedings paper. In: Editor,
F., Editor, S. (eds.) CONFERENCE 2016, LNCS, vol. 9999, pp. 1--13.
Springer, Heidelberg (2016).

\bibitem{ref_book1}
Author, F., Author, S., Author, T.: Book title. 2nd edn. Publisher,
Location (1999)

\bibitem{ref_proc1}
Author, A.-B.: Contribution title. In: 9th International Proceedings
on Proceedings, pp. 1--2. Publisher, Location (2010)

\bibitem{ref_url1}
LNCS Homepage, \url{http://www.springer.com/lncs}. Last accessed 4
Oct 2017
\end{thebibliography}
\end{document}
